% Options for packages loaded elsewhere
\PassOptionsToPackage{unicode}{hyperref}
\PassOptionsToPackage{hyphens}{url}
%
\documentclass[
  12pt,
]{article}
\usepackage{lmodern}
\usepackage{amssymb,amsmath}
\usepackage{ifxetex,ifluatex}
\ifnum 0\ifxetex 1\fi\ifluatex 1\fi=0 % if pdftex
  \usepackage[T1]{fontenc}
  \usepackage[utf8]{inputenc}
  \usepackage{textcomp} % provide euro and other symbols
\else % if luatex or xetex
  \usepackage{unicode-math}
  \defaultfontfeatures{Scale=MatchLowercase}
  \defaultfontfeatures[\rmfamily]{Ligatures=TeX,Scale=1}
\fi
% Use upquote if available, for straight quotes in verbatim environments
\IfFileExists{upquote.sty}{\usepackage{upquote}}{}
\IfFileExists{microtype.sty}{% use microtype if available
  \usepackage[]{microtype}
  \UseMicrotypeSet[protrusion]{basicmath} % disable protrusion for tt fonts
}{}
\makeatletter
\@ifundefined{KOMAClassName}{% if non-KOMA class
  \IfFileExists{parskip.sty}{%
    \usepackage{parskip}
  }{% else
    \setlength{\parindent}{0pt}
    \setlength{\parskip}{6pt plus 2pt minus 1pt}}
}{% if KOMA class
  \KOMAoptions{parskip=half}}
\makeatother
\usepackage{xcolor}
\IfFileExists{xurl.sty}{\usepackage{xurl}}{} % add URL line breaks if available
\IfFileExists{bookmark.sty}{\usepackage{bookmark}}{\usepackage{hyperref}}
\hypersetup{
  hidelinks,
  pdfcreator={LaTeX via pandoc}}
\urlstyle{same} % disable monospaced font for URLs
\usepackage[margin=1in]{geometry}
\usepackage{longtable,booktabs}
% Correct order of tables after \paragraph or \subparagraph
\usepackage{etoolbox}
\makeatletter
\patchcmd\longtable{\par}{\if@noskipsec\mbox{}\fi\par}{}{}
\makeatother
% Allow footnotes in longtable head/foot
\IfFileExists{footnotehyper.sty}{\usepackage{footnotehyper}}{\usepackage{footnote}}
\makesavenoteenv{longtable}
\usepackage{graphicx,grffile}
\makeatletter
\def\maxwidth{\ifdim\Gin@nat@width>\linewidth\linewidth\else\Gin@nat@width\fi}
\def\maxheight{\ifdim\Gin@nat@height>\textheight\textheight\else\Gin@nat@height\fi}
\makeatother
% Scale images if necessary, so that they will not overflow the page
% margins by default, and it is still possible to overwrite the defaults
% using explicit options in \includegraphics[width, height, ...]{}
\setkeys{Gin}{width=\maxwidth,height=\maxheight,keepaspectratio}
% Set default figure placement to htbp
\makeatletter
\def\fps@figure{htbp}
\makeatother
\setlength{\emergencystretch}{3em} % prevent overfull lines
\providecommand{\tightlist}{%
  \setlength{\itemsep}{0pt}\setlength{\parskip}{0pt}}
\setcounter{secnumdepth}{5}
\usepackage{amsthm}
\usepackage{amsmath}
\usepackage{amssymb}
\usepackage{verbatim}
\usepackage{hyperref}
\usepackage[T1]{fontenc}
\usepackage[utf8]{inputenc}
\usepackage[icelandic]{babel}
\hypersetup{colorlinks,allcolors=blue}
\usepackage{enumerate}
\usepackage{lastpage}
\usepackage[shortlabels]{enumitem}
\usepackage{fancyhdr}
\pagestyle{fancy}
\fancyhf{}
\fancyhead[R]{Gylfi Snær Sigurðsson}
\fancyhead[C]{Er hægt að mæla framfarir en ekki bara utanaðbókarlærdóm?}
\fancyhead[L]{Sumarverkefni}
\fancyfoot[L]{Háskóli Íslands}
\fancyfoot[R]{\thepage}
\usepackage{float}
\usepackage{algorithm}
\usepackage{algorithmicx}
\usepackage{algpseudocode}
\usepackage{caption}
\usepackage[sectionbib]{chapterbib}
\usepackage{booktabs}
\usepackage{longtable}
\usepackage{array}
\usepackage{multirow}
\usepackage{wrapfig}
\usepackage{float}
\usepackage{colortbl}
\usepackage{pdflscape}
\usepackage{tabu}
\usepackage{threeparttable}
\usepackage{threeparttablex}
\usepackage[normalem]{ulem}
\usepackage{makecell}
\usepackage{xcolor}

\author{}
\date{\vspace{-2.5em}}

\begin{document}

%\begin{titlepage}
%\thispagestyle{empty}
\begin{center}
\LARGE{\textbf{Sumarverkefni}}\\
\vspace*{2\baselineskip}
\Large{\textbf{Er hægt að mæla framfarir en ekki bara utanaðbókarlærdóm?}}
\end{center}
% \end{titlepage}
\thispagestyle{empty}
\newpage

\hypertarget{inngangur}{%
\section{Inngangur}\label{inngangur}}

Tutor-web er\ldots{} Gott að spyrja um góða leið til að útskýra það í byrjunn

Spurningar í tutor-web eru stundum búnar til þannig að kennari semur fyrst einn haus og svo tvo hauga af fullyrðingum, einn með réttum svörum og einn með röngum svörum. Svo eru spurningarnar búnar til með því að velja slembið rétt svar og slurk af röngum svörum (kannski með allt/ekkert ofangreint er rétt). Ef nemandi lærir ekki bara utanað, heldur er að auka skilning á verkefnið, þá ættu að sjást smátt og smátt framfarir í einkunn líka fyrir nýjar fullyrðingar.

Svo spurning kemur um að hvort það sé hægt að mæla framfarir, en ekki bara utanaðbókarlærdóm.

SKOÐA HVAÐ HÆGT SÉ AÐ BÆTA ÞEGAR SEINNA KEMUR.

\hypertarget{auxf0feruxf0}{%
\section{Aðferð}\label{auxf0feruxf0}}

\hypertarget{framkvuxe6md}{%
\subsection{framkvæmd}\label{framkvuxe6md}}

\hypertarget{uxfeuxe1ttakendur}{%
\subsection{Þáttakendur}\label{uxfeuxe1ttakendur}}

Úrtak rannsóknarinnar voru nemendur í líkindareikningur og tölfræði á vormisserinu 2020. Þar voru 294 nemendur sem svöruðu samtals 108.017 spurningum í heild.

\hypertarget{tuxf6lfruxe6uxf0ileg-uxfarvinnsla}{%
\section{Tölfræðileg úrvinnsla}\label{tuxf6lfruxe6uxf0ileg-uxfarvinnsla}}

\hypertarget{hugbuxfanauxf0ur}{%
\subsection{Hugbúnaður}\label{hugbuxfanauxf0ur}}

Öll tölfræðileg úrvinnsla, meðhöndlun gagna og líkanasmíð fór fram á forritunarmálinu
R v4.0.0 í viðmótinu RStudio v1.3.959

\hypertarget{guxf6gn}{%
\subsection{Gögn}\label{guxf6gn}}

\hypertarget{gagnahreinsun}{%
\subsection{Gagnahreinsun}\label{gagnahreinsun}}

\hypertarget{breytur}{%
\subsection{Breytur}\label{breytur}}

\hypertarget{auxf0feruxf0arfruxe6uxf0i-viuxf0-luxedkanasmiuxf0}{%
\subsection{Aðferðarfræði við líkanasmið}\label{auxf0feruxf0arfruxe6uxf0i-viuxf0-luxedkanasmiuxf0}}

\end{document}
